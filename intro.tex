\section*{Introduction} % Pas de numérotation
\addcontentsline{toc}{section}{Introduction} % Ajout dans la table des matières

Le présent rapport est une présentation du travail effectué dans le cadre de mon stage de fin d'études à la direction des systèmes d'information GM-Global Markets Distribution EBusiness de Natixis.

Contexte : 

Le Big Data permet de traiter les données non-structurées encore très peu exploitées pour catégoriser les clients qui traitent avec Natixis sur le périmètre des produits de change (Forex). Dans ce contexte, Natixis souhaite travailler dans le cadre d’un stage de fin d’études
sur l’analyse et la recherche d’informations des données récoltées au travers deux streams de données :

\begin {itemize}
\item Le premier Stream de données est axé sur les informations des clients de Natixis et sur les opérations d’achat/vente qu’ils effectuent avec Natixis.

\item Le deuxième Stream de données concerne les informations marché (le carnet de liquidité et les informations financières).
\end{itemize}

Pour réaliser ce travail, j'ai commencé par l'élaboration d'une étude approfondie des technologies Big Data et son ecosystem, du spark, des bases de données NOsql, du data mining et de machine learning.

Ensuite a travers cette étude j'ai mis en place d'une architecture Big data. Pendant ce travail j'ai utiliser une machine virtuelle pour faire les differents test.
